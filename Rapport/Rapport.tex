\newcommand{\auteur}{Stalder Lawrence, Pombo Dias Miguel, Cotza Andrea Roberto, \\
	
	Verdasca Jimmy, Clavier Tony \& Guillod Maxime}
\newcommand{\cours}{GEN}
\newcommand{\ecole}{IL --- TIC --- HEIG-VD}
\newcommand{\domaine}{PDG}
\newcommand{\titre}{WordOn Desktop}

\documentclass[a4paper,12pt]{article}
%
\author{\auteur}
\title{\titre}
\date{\today}

\usepackage[frenchb]{babel}
\usepackage{fancyhdr}
\usepackage{graphicx}
\usepackage{amsmath}
\usepackage{listingsutf8}
\usepackage{color}
\usepackage{enumerate}
\usepackage[utf8]{inputenc}
\usepackage[T1]{fontenc}
\usepackage{float}
\usepackage{geometry}
\usepackage{amssymb,mathtools,pifont}
\usepackage{enumitem}
\usepackage{xspace}
\usepackage{appendix}
\usepackage{pdfpages}
% Liens
\usepackage[hyphens]{url}
\usepackage{hyperref}
\geometry{verbose,tmargin=2.5cm,bmargin=2.8cm,lmargin=1.8cm,rmargin=1.8cm}
\selectlanguage{frenchb}
\frenchbsetup{StandardLists=true}
\DeclareGraphicsExtensions{.pdf,.png,.jpg}
\setlength\parindent{0pt}
\setlength{\parskip}{0.7em}

\usepackage{listings}
\lstset{
	breaklines=true, 
	basicstyle=\scriptsize,
	inputencoding=utf8/latin1,
	extendedchars=true,
	numbers=left,
	firstnumber=1,
	numberfirstline=true, 
	language=Java,
	keywordstyle=\color{blue}\ttfamily,
	stringstyle=\color{red}\ttfamily,
	commentstyle=\color{black}\ttfamily
}


% headers & footers
\pagestyle{fancy}

\lhead{\domaine}
\rhead{\titre\space\includegraphics[scale=0.12]{logo/logo.png}}

\renewcommand{\footrulewidth}{0.4pt}% default is 0pt
\lfoot{\auteur}
\cfoot{}
\rfoot{\thepage}

%%%%%%%%%%%%%%%%%%%%%%%%%%%%%%%%%%%%%%%
%%%%%%% BEGIN DOCUMENT
%%%%%%%%%%%%%%%%%%%%%%%%%%%%%%%%%%%%%%%

\begin{document}
	\clearpage
	\maketitle
	\thispagestyle{empty}
	
	\maketitle
	\begin{figure}[h!]
		\centering
		\includegraphics[scale=1]{logo/logo.png}
	\end{figure}
	\newpage
	
	% % Entete première page
	% \thispagestyle{empty}
	% %
	% \noindent \cours \hfill \ecole{} \newline
	% \noindent \auteur \hfill \today \newline
	% \hrule
	% \vspace{7mm}
	% \noindent {\large \bf \domaine } \hfill \titre {\large \bf }\\[3mm]
	% \hrule
	
	\tableofcontents
	
	\listoffigures
	
	% On a pas de tableau
	% \listoftables
	
	\newpage
	
	\section{Description générale du projet}
	\subsection{Cadre / contexte}
	Ce projet a lieu dans le cadre du cours PDG à la HEIG-VD à Yverdon-les-Bains dans le cadre de notre dernière année de bachelor en informatique.
	
	\subsection{But(s) visé(s)}
	A l'issue de ce cours ainsi que de ce projet, nous devrons être capables de spécifier, coder ainsi que tester une application de taille importante.
	
	De plus, ce projet nous apprend à acquérir par nous-mêmes des connaissances sur des nouveaux sujets en fonction de nos besoins lors de la création de notre application. 
	
	Il y a également toute la gestion de la problématique d'un projet en équipe, en groupe de 6 personnes. En effet, ceci demande une très bonne coordination entre chaque membre afin de bien séparer le travail et de faire en sorte d'avoir un projet unique et harmonieux à terme. Ceci nous demandera d'utiliser plusieurs outils spécifiques.
	
	\section{Description du jeu sur Smartphone}
		\subsection{Fonctionnalités offertes}
			\subsubsection{Inscription}
			Le jeu est accessible soit de manière anonyme, à travers un compte invité qui réduit l'utilisation de certaines fonctionnalités notamment la recherche d'ami et le chat, soit en s'inscrivant au jeu via Facebook ou via une adresse e-mail ce qui nous permet de profiter du jeu complet.
			Lors de la première utilisation de l'application, un tutoriel est jouable afin de comprendre les règles et les subtilités du jeu.
			\subsubsection{Mode normal}
			Le principal mode du jeu est celui du mode normal qui est accessible		depuis l'onglet parties, en commençant une nouvelle partie soit contre
			\begin{itemize}
				\item un ami
				\item un adversaire aléatoire
				\item un adversaire récent
			\end{itemize}
			Ou sur invitation d'un autre joueur.
			Une partie démarre ainsi et l'on peut commencer à jouer.
			\subsubsection{Mode tournoi}
			Le mode tournoi est plus particulier. On y accède via l'onglet tournoi en rejoignant un tournoi compétitif (contre des adversaire aléatoire) ou contre des amis ou toujours sur invitation d'un ami.
			Le mode nous permet de jouer contre 5 adversaire sur 5 jours en mode round robin, avec à la fin un classement parmi les participants et des récompense en fonction de la performance.
			\subsubsection{Bonus / effet}
			Les cases disponible pour jouer un mot contiennent parfois des bonus ou effets, ceux-ci comprennent .
			\begin{itemize}
				\item des bonus multiplicateur : la lettre joué sur cette case verra sa valeur doublé ou triplé
				\item des cases "donner sa lettre à son adversaire" où comme son nom l'indique la lettre joué sur cette case sera donnée en bonus à son adversaire.
			\end{itemize}
		
		\subsection{Règles du jeu}
			\subsubsection{Partie}
			Pour gagner une partie les joueurs doivent rassembler le plus de points possible et se base sur une variante d'un jeu de Scrabble.
			En effet, au commencement avec 114 lettres sont disponible. Comme au Scrabble, un lettre a une valeur attribuée. \newline
			 Chaque joueur pioche jusqu'à avoir 7 lettres au début de la partie ainsi qu'à la fin de chaque tour. Un joueur est désigné au hasard pour débuter le jeu. 2 nouvelles lettres viendront s'ajouter et seront considéré comme optionnelle. Devant lui, 7 cases sont affichées avec pour certaines d'entre elles des bonus ou effet.
			[Insert images]
			Il peut 
			\begin{itemize}
				\item jouer un mot valide (du dictionnaire) de 2 à 7 lettres.
				\item échanger des lettres de sa main et passer son tour.
			\end{itemize}
			Les points sont compté en additionnant la valeur de toutes les lettres jouées avec leur bonus avec ces règles : 
			\begin{itemize}
				\item si des lettres optionnelles ne sont pas toutes jouées, la valeur des lettres non-jouées est déduite des points.
				\item si un mot de 7 lettres est joué les points du mots sont doublé
			\end{itemize}
			Le tour passe à l'autre joueur et la partie continue ainsi de suite.
			La partie se termine lorsqu'il n'y plus de lettre disponibles et que plus de mots peuvent être formées par les joueurs ou qu'il passent tous les deux leur tour. 
			Un joueur est absent plus de 72h, c'est à dire qu'il ne joue pas de mot, perds automatiquement la partie.
			\subsubsection{Mode tournoi}
			5 personnes participent à un tournoi en plus de soi-même. Une partie est limitée à 24h et le tournoi dure 5 jours. Chaque jour nous affrontons un adversaire différent, et au terme du tournois les 5 adversaires auront été rencontré. Les points de toutes les parties sont additionnées pour nous donner les points gagnés du tournoi. À partir de ces points les classement est effectué.
			Si une personne ne joue pas durant 24h ou si une partie n'est pas terminée, les points réalisé sont reporté comme point de partie.
			
	
	\section{Jeu sur PC}
	\subsection{Introduction}
	Nous avons du optimiser principalement l'interface afin de correspondre aux spécifications d'un ordinateur.\\
	Pour la partie logique, nous avons essayé de coller au mieux à l'application afin de pouvoir offrir une expérince utilisateurs similaire que sur la version mobile. 
	
	\subsection{Interface graphique}
	L'interface utilisateur à été revue et optimisée pour l'affichage sur ordinateur en raison de la taille d'écran ainsi que sa résolution plus importante. \\
	Cependant, nous avons voulu garder la patte graphique du jeu original sur mobile en reprenant les codes couleurs et les éléments visuels spécifique au jeu.
	
		\subsubsection{Connexion et inscription}
		Les fenêtres de connexion (\textit{SignIn}) ainsi que celle de l'enregistrement (\textit{SignUp}) sont épurée afin de n'afficher que les informations utiles.
		
		\begin{figure}[h]
			\centering
			\includegraphics[width=0.4\linewidth]{img/signin.jpg}
			\caption{Fenêtre de connexion}
		\end{figure}
	
		\begin{figure}[h]
			\centering
			\includegraphics[width=0.4\linewidth]{img/signup.jpg}
			\caption{Fenêtre d'enregistrement}
		\end{figure}
		
		\subsubsection{Jeu}
		La taille plus importante de l'écran nous permet nottament d'afficher le lobby, la partie séléctionnée ainsi que le chat correspondant à la partie en cours. Ceci rend l'expérience de jeu plus adaptée à l'environnement desktop et permet une meilleure vue de toutes les parties en cours.
		
		\begin{figure}[h]
			\centering
			\includegraphics[width=0.6\linewidth]{img/main.jpg}
			\caption{Interface principale}
		\end{figure}
		
	\subsection{Conception}
	
	\subsection{Implémentation}
		\subsubsection{Dictionnaire}
		Nous avons effectué du reverse engineering\footnote{\textbf{Reverse engineering} :  \href{https://fr.wikipedia.org/wiki/R\%C3\%A9tro-ing\%C3\%A9nierie}{Définition Wikipédia}} afin de comprendre le fonctionnement internet du jeu.\\
		Après avoir enregistré plusieurs fois les lettres sorties par l'application mobile, nous avons pu constater et mettre en avant que le dictionnaire (les lettres tirées) sont définie par rapport à stack prédéfinit qui est simplement mélangé au début de la partie. En effet, lors d'une partie, on trouvera toujours le même nombre de fois le \textit{E} qui sort par exemple.
		
		Nous sommes arrivé au dictionnaire suivante : 
		
		\begin{table}[h]
			\centering
			\begin{tabular}{cc}
				\textbf{Lettre} & \textbf{Nombre} \\
				A     & 12 \\
				B     & 2 \\
				C     & 3 \\
				D     & 3 \\
				E     & 17 \\
				F     & 2 \\
				G     & 2 \\
				H     & 2 \\
				I     & 10 \\
				J     & 1 \\
				K     & 1 \\
				L     & 5 \\
				M     & 3 \\
				N     & 6 \\
				O     & 7 \\
				P     & 3 \\
				Q     & 1 \\
				R     & 7 \\
				S     & 8 \\
				T     & 6 \\
				U     & 8 \\
				V     & 1 \\
				W     & 1 \\
				X     & 1 \\
				Y     & 1 \\
				Z     & 1 \\
				\textit{JOKER} & 2 
			\end{tabular}
			\caption{Nombre d'occurence de chaque lettre dans notre dictionnaire}
		\end{table}
		
		\subsubsection{Algorithme tirage aléatoire}
		Suite à la mise en avant du dictionnaire, soit du stack définit de lettre qui peuvent sortir au cours d'une partie, nous avons analysé leur ordre de sortie en essayant de mettre un avant un paterne, ou toutes autres logique. \\
		Nous avons constaté que l'ordre d'arrivée des lettres est purement aléatoire malgré les apparences. En effet, au fur et à mesure que l'on avance dans une partie, on va acumuler des lettres peut utilisées de la langue française, telle que le \textit{J}, \textit{K}, \textit{V}, \textit{W}, \textit{X}, etc, ce qui rend un apparence seulement, la partie plus dur lorsque nous arrivons au terme.
		
		\subsubsection{Communication}
	
	\subsection{Tests}
	\subsubsection{Bug connu}
	
	\subsubsection{Fonctionnalité pouvant être ajoutée}
	
	\section{Technologie utilisée}
	Nous avons décidé de coder et développer notre application sans framework. En effet, aucun des frameworks connus ne nous aurait réellement aidés par rapport à nos beosin. De plus, prendre du temps pour comprendre et savoir utiliser correctement un ou plusieurs nouveaux frameworks aurait pû nous prendre beaucoup de temps. \\
	C'est pour ces raisons que nous choisis de coder principalement notre projet en pure Java, y compris pour toute la partie communication. 
	
		\subsection{JUnit}
		JUnit est un framework Java permettant de faire des tests unitaires. Ce dernier comporte de nombreuses méthodes afin d'automatiser les tests, et est utilisé par \hyperref[maven]{Maven}.\\
		Il permet par exemple de comparer des valeurs attendues de méthode avec ce que l'on reçoit vraiment. On peut tester l'intégralité des communications avec un tel système. De plus, il permet de s'assurer que même lors du développement d'un tel projet, les méthodes et fonctions codées précédemment n'ont pas un comportement différent en fonction du contexte d'appel. 
		
		\subsection{JavaFX}
		JavaFX est une bibliothèque d'interface graphique. Nous l'avons utilisée, avec l'aide de SceneBuilder\footnote{\textbf{SceneBuilder} : \href{http://gluonhq.com/products/scene-builder/}{gluonhq.com/products/scene-builder/}} qui permet de créer des interfaces graphiques de manière intuitive et visuelle. \\
		Cette librairie permet également de gérer tout ce qui est des interactions avec notre environnement graphique à l'aide de \textit{listener} qu'on peut lier à nos briques, nos éléments graphiques. 
		
		\subsection{GitHub}
		GitHub\footnote{\textbf{GitGub} : \href{http:/github.com}{github.com}} utilise la technologie git qui permet de gérer les fichiers d'un projet sur différentes branches, tout en faisant du versionning en ligne. Il offre également la possibilité de gérer les issues avec un système communautaire (du groupe) pour la résolution, mais également une gestion de planning et d'attribution de tâches. \\
		Ceci nous a permis de bien pouvoir séparer le travail et les parties, pour ensuite, une fois validé par les tests, les intégrer dans notre projet, soit sur la branche master. 
		
		\subsection{Apache Maven} \label{maven}
		Maven est un outil de gestion et d'automatisation de projet Java. Il permet notamment d'automatiser les dépendances de notre projet en téléchargent automatiquement les librairies utilisées par les autres membres du groupe. Il permet également de tester le projet en lançant tout les tests de notre projet avant de le compiler afin de garantir du bon fonctionnement de ce dernier. 
		
		\subsection{Docker}
	
	\section{Conclusion}
	
	\newpage
	\section{Annexes}

	\includepdf[scale=0.95, pagecommand={}]{journalTravail/vide.pdf}
	
	
	
\end{document}

